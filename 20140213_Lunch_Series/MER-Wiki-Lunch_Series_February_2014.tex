\documentclass{beamer}
%\documentclass[handout]{beamer}

\usepackage{color}
\usepackage[thicklines]{cancel}


\renewcommand{\CancelColor}{\color{red}}

%This removes an error message.
\let\Tiny=\tiny

\newcommand{\bF}{{\mathbb{F}}}
\newcommand{\bQ}{{\mathbb{Q}}}
\newcommand{\bZ}{{\mathbb{Z}}}

\renewcommand{\P}{{\mathcal{P}}}

\newcommand{\tor}{\textnormal{tor}}

\renewcommand{\mod}[1]{{\ifmmode\text{\rm\ (mod~$#1$)}\else\discretionary{}{}{\hbox{ }}\rm(mod~$#1$)\fi}}
\newcommand{\legendre}[2]{\genfrac{(}{)}{}{}{#1}{#2}}

%try beamerthemebars or beamerthemeclassic
\usepackage{beamerthemesplit}
\usetheme{boxes}
%\usepackage{beamertemplates}
\setbeamertemplate{navigation symbols}{}

\title[The Math Exam Resource Wiki]{The Math \xcancel{Exam} Educational Resources Wiki \\ UBC Lunch Series for Teaching and Learning}
\author{Carmen Bruni
\newline
On behalf of Christina Koch, Bernhard Konrad, Iain Moyles and the entire MER wiki team}
\institute{University of British Columbia}
%\date{February 6th, 2014}

\date{February 13th, 2014 \\
\includegraphics[scale=0.4]{Math_kid.jpg}
}




\beamertemplateshadingbackground{white}{white}%{yellow!50}{magenta!50}


\hypersetup{colorlinks=true, linkcolor=cyan, urlcolor=cyan}
\begin{document}

\frame{\titlepage}

%\section[Outline]{}

%\frame{\tableofcontents}

%%\section[Introduction]{}

%\frame{\tableofcontentscurrent}

%\subsection{Introduction}

\frame
{
  \begin{block}{Outline}

      \begin{itemize}
				\item Welcome and brief history.
                \item Recent updates.
				\item Current ideas.
                \item Current issues.
				\item Discussion: future directions of the MER wiki.
      \end{itemize}
  \end{block}

  \begin{block}{Items to ponder during the short demos}

      \begin{itemize}
				\item Is this work sustainable? How can we make it so?
				\item How one can use this resource in class/for instructors?
				\item Where do we, as a department, want to go with this resource?
                \item How do we best align this resource with our teaching goals?
      \end{itemize}
  \end{block}

}


\frame
{
  \begin{block}{Brief history of the MER wiki}

      \begin{itemize}
				\item Project started \href{http://wiki.ubc.ca/index.php?title=Science:Math_Exam_Resources&oldid=139951}{20 Feb 2012.} Now, it has become \href{http://wiki.ubc.ca/Science:Math_Exam_Resources}{this!}

                \item Goal: improve quality of content, system of content delivery.

				\item Wiki format allows flat hirarchy, is easy for new contributors, great to work collaboratively and is intuitive to use.
      \end{itemize}
    \end{block}

    \begin{block}{How far did we get in two years?}
        \begin{itemize}
            \item 39 complete exams, 954 fully written solutions with hints (and counting!) by about 35 contributors.

            %\item \href{nbviewer.ipython.org/github/MER-wiki/google-analytics/blob/master/mer_py_google_analytics.ipynb}{Do students use it? Boy do they!}
         \end{itemize}
      \end{block}
}

\frame
{
    \frametitle{Student feedback}
Do students use it? \href{http://htmlpreview.github.io/?https://github.com/MER-wiki/google-analytics/blob/master/presentations/2014-02-13_Lunch_Series/output-hidden.html}{Oh yes, they do!}

\begin{itemize}
\item ``I just wanted to express my sincere gratitude for the resources you have provided us! This first year course has certainly been a challenge for many of us, and your help really means a lot to us. Indeed, creating this resource must be a crystal of hard work, effort
and time, that your team have contributed, for the sake of others.''
\item `` I used the tagging system and found it to be very helpful as I could find similar types of questions that I had a difficult time with quickly (sort of like using an index in a textbook).''
\item ``The tagging system was GREAT! It's funny because I remember seeing those tags for some topics but not the one I was looking for, so I was frustrated. Then the next day it was there! That was cool.''
\end{itemize}

}


\frame
{
  \frametitle{What happened this past term?}

      \begin{itemize}
            %\item Insight on usage and exam design from  \href{nbviewer.ipython.org/github/MER-wiki/google-analytics/blob/master/mer_py_google_analytics.ipynb}{google analytics}
            \item Additional insight on usage and exam design from  \href{http://wiki.ubc.ca/Science:Math_Exam_Resources/Courses/MATH105/April_2010/Question_6}{the rating bar.} We can now start to rate by \href{http://wiki.ubc.ca/Sandbox:MER-rating}{difficulty.}
            %\item Regular meetings with CTLT staff (wiki admins).
            \item Improved the \href{http://wiki.ubc.ca/Category:MER_Tag_Linear_approximation}{tagging system} and connect it to the  \href{http://www.math.ubc.ca/~cbruni/pdfs/Math103JanApr2014/syllabus.pdf}{course syllabus}. Show syllabus on \href{http://wiki.ubc.ca/Science:Math_Exam_Resources/Courses/MATH103}{exam course page}.
            %\item {\color{red} I THINK THIS WOULD TO MENTION. Math 110 pilot project.}
            \item Display and remix wiki content on \href{http://wordpress.org/plugins/wiki-embed/}{wordpress} and \href{http://wiki.ubc.ca/Sandbox:MER_Wiki_to_LMS}{connect}.
            \item Grant applications to \href{http://tlef.ubc.ca/tlef-criteria/}{Teaching and Learning Enhancement Fund} and \href{http://wiki.ubc.ca/Library:Scholarly_Communications/Innovative_Dissemination_of_Research_Award}{Innovative Dissemination of Research Award}.

\end{itemize}
}




\frame
{
  \frametitle{Research on worked-examples}

\alert{Question:} Is this useful for students or harmful?\\

      \begin{itemize}
            \item For the sake of this talk, let's assume that our goal is to aid students on achieving a high score on their examinations.
            \item Research suggests that
            \begin{enumerate}
            \item Students who see worked examples and conventional problems versus only seeing conventional problems do better on test questions containing similar problems (Sweller-Cooper '85), (Cooper-Sweller '87), (Paas-Van Merri\"enboer '94).
            \item Students also spend less time on worked examples than on conventional problems (S-C '85), (C-S '87), (P-V M '94).
            \end{enumerate}
      \end{itemize}
}

\frame
{
  \frametitle{Research on worked-examples}

            \begin{enumerate}
            \setcounter{enumi}{2}
            \item Students who are in the worked example group also can perform better and spend less time on transfer problems (that is, examples not identical to the practice ones) provided the variance in difficulty is not too large (C-S '87), (P-V M '94).
            \item Strong students are not effected by the type of practice so long as they spend ``enough time'' on task. Weak students however perform much better on tests provided they have spent enough time on worked examples  (C-S '87).
            \end{enumerate}
}

\frame
{
  \frametitle{Research on worked-examples}

            \begin{enumerate}
            \setcounter{enumi}{4}
            \item When normalized for time, students perform substantially better and are faster on transfer problems than a conventional problem group (C-S '87).
            \item Students seeing worked examples on problems of large variability will in general perform better on transfer tests (P-V M '94). \pause
            \end{enumerate}

          \alert{Summary: Students need to develop schemata for organizing problems which can be faciliated by worked examples.}
}

\frame
{
  \frametitle{Where do we want to take this?}

\begin{block}{Tie to our teaching goals}
      \begin{itemize}
            \item Simply continue to post more exam solutions?
            \item Add non-exam questions and ``fill in the details'' solutions? %Eg integrate open textbooks, advertise for tutors, MLC hours.
            \item Integrate into courses: Some percentage of grade to invite students to contribute. For example, explain a concept and post on tag pages.
            \item \alert{Wiki's vision:} Make this the best resource possible for undergraduate students at UBC.
      \end{itemize}
\end{block}

%\begin{block}{Delivery}
%        \begin{itemize}
%            \item openEducation: \href{learningpod.com}{learningpod}, %\href{mathelike.de}{Mathelike}, \href{https://www.myopenmath.com/}{My open math}.
%            \item Kickstarter to develop new features?
%        \end{itemize}
%\end{block}


\begin{block}{Be part of the math department vision}
        \begin{itemize}
            \item How the wiki can make an instructor's life easier?
            \item How can we make it a more year-round resource?
            \item How can we increase sustainability?
            %\item Tie to UBC \href{http://strategicplan.ubc.ca/}{mission and goals}.
        \end{itemize}
\end{block}

}

\frame
{
  \frametitle{Ways to Get Involved}

      \begin{itemize}
				\item Write hints, explanations, solutions.
                \item Send us exam solutions.
                \item If you're teaching, advertise to your students or even tie to your course.
                %\item Interpret and write (python) code to visualize google analytics.
                \item Code away on our templates in MediaWiki.
                \item Give feedback! What do you like or dislike? What should be added or removed?
                \item Send us your ideas! Email us at  mer-wiki@math.ubc.ca
                %\item proof-read grant proposals
                %\item Join the \href{hojoki.com}{hojoki group} to stay up to date and streamline tasks.
      \end{itemize}

      \bigskip

\begin{center}
\pause \alert{Thank you!}
\end{center}
}


%\begin{frame}
%\frametitle{Cryptography Scheme}
%Here is another slide.
%\end{frame}

\end{document}

%sagemathcloud={"zoom_width":105}